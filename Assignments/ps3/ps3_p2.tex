\documentclass[12pt]{article}%
\usepackage{amsfonts}
\usepackage{fancyhdr}
\usepackage[hidelinks]{hyperref}
\usepackage[a4paper, top=2.5cm, bottom=2.5cm, left=2.2cm, right=2.2cm]%
{geometry}
\usepackage{times}
\usepackage{amsmath}
\usepackage{amsthm}
\usepackage{changepage}
\usepackage{amssymb}
\let\proof\relax
\let\endproof\relax
\usepackage{graphicx}%
\setcounter{MaxMatrixCols}{30}
\newtheorem{theorem}{Theorem}
\newtheorem{corollary}[theorem]{Corollary}
\newtheorem{definition}[theorem]{Definition}
\newtheorem{lemma}[theorem]{Lemma}
\newtheorem{proposition}[theorem]{Proposition}
\newenvironment{proof}[1][Proof]{\textbf{#1.} }{\ \rule{0.5em}{0.5em}}

\begin{document}

\title{COS 445 - PSet 3, Problem 2} %Replace X with homework number, Y with problem number.
\author{Anakin Skywalker} %Write your name here
\date{\today}
\maketitle
\section*{Problem 2: Limited Information Scoring Rules}
\subsection*{Part a}

I don't know.

\subsection*{Part b}

We wish to design a proper scoring rule which takes as input only the reported mean of $X$, and strictly incentivizes the predictor to report the true mean of $X$.

Let us define the scoring function $S(y, x) = -(y - x)^2$. To find the unique argmaximum, we compute the derivative of $\mathbb{E}[S(y, x)]$:

\begin{equation}
\begin{split}
0 &= \mathbb{E}[-2(y - x)]\\
0 &= -2(y - \mathbb{E}[x])\\
0 &= y - \mathbb{E}[x]\\
y &= \mathbb{E}[x]
\end{split}
\end{equation}

Therefore, the unique argmaximum is $y = \mathbb{E}[x]$. We note that $\mathbb{E}[S]$ is taken by linearity of expectation, with $y$ not being a random variable (and thus its expectation is always itself). If this is proper, than $y = \mathbb{E}[x]$ must be the global max. We know that $S$ is strictly increasing on the left side, because its derivative $-2(y - \mathbb{E}[x])$ is positive for all $y < \mathbb{E}[x]$. Likewise, $S$ is strictly decreasing on the right side because $-2(y - \mathbb{E}[x])$ is negative for all $y > \mathbb{E}[x]$. Therefore, $y = \mathbb{E}[x]$ is the global max and so $S$ is a proper scoring rule.

\subsection*{Part c}

We wish to prove that there does not exist any scoring rule $S$ which takes only as input the reported variance of $X$ that strictly incentivizes the predictor to report the true variance of $X$.

Let us assume for contradiction that such an $S$ existed. We consider two random variables $X$ and $Y$ with the same variance, and a third distribution $Z$ which is equal to $X$ with probability $\frac{1}{2}$ and $Y$ with probability $\frac{1}{2}$. We set the value of $X = 1$ with variance 0, and $Y = 2$ with variance 0. In other words, $X$ and $Y$ will be always be a constant value. We also assume that the scoring rule works for $X$ and $Y$.

We then compute the unique argmaximum $\mathbb{E}[S(y, z)]$. By linearity of expectation, this can be related to the expectations of $Y$ and $Z$. We also note that because the variance of $X$ and $Y$ is 0, taking the argmax for each one will also be 0 since the scoring function incentivizes the predictor to report the true variance. Since $Z$ is equal to $X$ with probability $\frac{1}{2}$ and $Y$ with probability $\frac{1}{2}$, we compute the argmaximum as follows:

\begin{equation}
\mathbb{E}[S(y, z)] = 0.5 \cdot 0 + 0.5 \cdot 0 = 0
\end{equation}

However, this is a contradiction because the true variance of $Z$ is in fact $\frac{1}{2}$. Therefore, it is not possible to create a scoring rule that incentivizes reporting the true variance when given only the reported variance as input.

\end{document}