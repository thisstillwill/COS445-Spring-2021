\documentclass[12pt]{article}%
\usepackage{amsfonts}
\usepackage{fancyhdr}
\usepackage[hidelinks]{hyperref}
\usepackage[a4paper, top=2.5cm, bottom=2.5cm, left=2.2cm, right=2.2cm]%
{geometry}
\usepackage{times}
\usepackage{amsmath}
\usepackage{amsthm}
\usepackage{changepage}
\usepackage{amssymb}
\let\proof\relax
\let\endproof\relax
\usepackage{graphicx}%
\setcounter{MaxMatrixCols}{30}
\newtheorem{theorem}{Theorem}
\newtheorem{corollary}[theorem]{Corollary}
\newtheorem{definition}[theorem]{Definition}
\newtheorem{lemma}[theorem]{Lemma}
\newtheorem{proposition}[theorem]{Proposition}
\newenvironment{proof}[1][Proof]{\textbf{#1.} }{\ \rule{0.5em}{0.5em}}

\begin{document}

\title{COS 445 - PSet 1, Problem 2} %Replace X with homework number, Y with problem number.
\author{Mark Watney} %Write your name here
\date{\today}
\maketitle
\section*{Problem 2: Both Sides Propose}
We wish to either prove that Both-Proposing Deferred Acceptance always terminates in a stable matching or provide an example of preferences and order of proposals such that BPDA does not output a stable matching.

Let there be two students, $S_1$ and $S_2$, and two universities $U_1$ and $U_2$. Let their preferences be the following:

\begin{center}
\begin{tabular}{|c|c|}
\hline
Element & Preferences \\ \hline
$S_1$   & $U_1 > U_2$ \\ \hline
$S_2$   & $U_1 > U_2$ \\ \hline
$U_1$   & $S_1 > S_2$ \\ \hline
$U_2$   & $S_1 > S_2$ \\ \hline
\end{tabular}
\end{center}

Since students propose first, let $S_2$ begin and propose to $U_1$, forming a pair. The remaining unmatched university, $U_2$, then proposes to the remaining student $S_1$ because they are their first choice. The algorithm terminates, since there are no more unmatched students. However, the resulting matching $(S_2, U_1)$ and $(S_1, U_2)$ is not stable. Both $S_1$ and $U_2$ prefer each other (in fact they are each other's first choice) over the matching they got, meaning $(S_1, U_1)$ is a blocking pair and would be strictly happier together. Therefore, BPDA does not always terminate in a stable matching.

\end{document}