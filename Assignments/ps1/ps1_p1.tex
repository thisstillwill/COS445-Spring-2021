\documentclass[12pt]{article}%
\usepackage{amsfonts}
\usepackage{fancyhdr}
\usepackage[hidelinks]{hyperref}
\usepackage[a4paper, top=2.5cm, bottom=2.5cm, left=2.2cm, right=2.2cm]%
{geometry}
\usepackage{times}
\usepackage{amsmath}
\usepackage{amsthm}
\usepackage{changepage}
\usepackage{amssymb}
\let\proof\relax
\let\endproof\relax
\usepackage{graphicx}%
\setcounter{MaxMatrixCols}{30}
\newtheorem{theorem}{Theorem}
\newtheorem{corollary}[theorem]{Corollary}
\newtheorem{definition}[theorem]{Definition}
\newtheorem{lemma}[theorem]{Lemma}
\newtheorem{proposition}[theorem]{Proposition}
\newenvironment{proof}[1][Proof]{\textbf{#1.} }{\ \rule{0.5em}{0.5em}}

\begin{document}

\title{COS 445 - PSet 1, Problem 1} %Replace X with homework number, Y with problem number.
\author{Mark Watney} %Write your name here
\date{\today}
\maketitle
\section*{Problem 1: Instances with many stable matchings}
We wish to prove that for all even $n \ge 2$ there exists a stable matching instance with $n$ students and $n$ universities with one slot each, such that there are at least $2^{\frac{n}{2}}$ distinct stable matchings.

We start by proving the claim for the base case. When $n = 2$ there are only two students, $S_1$ and $S_2$, and two universities $U_1$ and $U_2$. Let us assign the preferences so that $S_1$ prefers $U_1$ to $U_2$, $S_2$ prefers $U_2$ to $U_1$, $U_1$ prefers $S_2$ to $S_1$, and $U_2$ prefers $S_1$ to $S_2$. This means there are only two distinct stable matchings. Either the pairs are $(S_1, U_1)$ and $(S_2, U_2)$ or they are $(S_1, U_2)$ and $(S_2, U_1)$. The two matchings are stable, because changing one of them would not strictly enhance the happiness of all elements in the new pairs. In the first possible matching, for example, one might change the pairings such that $U_2$ gets its first choice ($S_1$) and $U_1$ gets its first choice ($S_2$). However, this would mean that the happiness of $S_1$ would decrease since $U_1$ was their first choice and was originally paired with them. Since there are $2^{\frac{2}{2}} = 2$ distinct stable matchings when $n = 2$, we have proven the claim for the base case. 

We will then prove the claim for all even $k \ge 2$ through induction, by assuming that the claim is true for $k \ge 2$. It remains to be shown that the claim holds for $k + 2$. 
We can consider a pair of students $(S_i, S_{i + 1})$ and a pair of universities $(U_i, U_{i + 1})$ in sequence. Let us assign the preferences to mirror those seen when $n = 2$:

\begin{center}
\begin{tabular}{|c|c|}
\hline
Element     & Preferences       \\ \hline
$S_i$       & $U_i > U_{i+1}$   \\ \hline
$S_{i + 1}$ & $U_{i+1} > U_i$   \\ \hline
$U_i$       & $S_{i + 1} > S_i$ \\ \hline
$U_{i+1}$   & $S_i > S_{i + 1}$ \\ \hline
\end{tabular}
\end{center}

This means that, as in the base case, there will never be a blocking pair that could exist between $S_i$, $S_{i + 1}$, $U_i$ and $U_{i + 1}$. If each pair of students and each pair of universities is considered in sequence, there are $\frac{n}{2}$ total pairs (since $n$ is always even). However, it might still be possible to have blocking pairs formed between students and universities that are not in sequence. To prevent this, let $S_i$, $S_{i + 1}$, $U_i$ and $U_{i + 1}$ prefer each each other more than the other $n - 2$ students and universities. This way, the matchings formed will always follow the pattern seen in the base case. Therefore, there must be at least $2^{\frac{n}{2}}$ distinct stable matchings for all even $n \ge 2$.

\end{document}