\documentclass[12pt]{article}%
\usepackage{amsfonts}
\usepackage{fancyhdr}
\usepackage[hidelinks]{hyperref}
\usepackage[a4paper, top=2.5cm, bottom=2.5cm, left=2.2cm, right=2.2cm]%
{geometry}
\usepackage{times}
\usepackage{amsmath}
\usepackage{amsthm}
\usepackage{changepage}
\usepackage{amssymb}
\let\proof\relax
\let\endproof\relax
\usepackage{graphicx}%
\setcounter{MaxMatrixCols}{30}
\newtheorem{theorem}{Theorem}
\newtheorem{corollary}[theorem]{Corollary}
\newtheorem{definition}[theorem]{Definition}
\newtheorem{lemma}[theorem]{Lemma}
\newtheorem{proposition}[theorem]{Proposition}
\newenvironment{proof}[1][Proof]{\textbf{#1.} }{\ \rule{0.5em}{0.5em}}

\begin{document}

\title{COS 445 - PSet 4, Problem 2} %Replace X with homework number, Y with problem number.
\author{Bro} %Write your name here
\date{\today}
\maketitle
\section*{Problem 2: Noisy Optimizers aren’t Good Enough}
\subsection*{Part a}
We wish to prove that for any given $\vec{b}_{-i}$ there exists a price $p$ such that bidder $i$ will either win the item and pay $p$ or not get the item (and pay nothing) no matter what bid they make.

Suppose the auction is run with $\vec{b}_{-1}$, that is without bidder $i$ being considered. Let $p$ be the bid of the resulting winner $j \neq i$. It is already given that each accepted bid is nonnegative. We note that when using $A(\vec{b}_{-1}, -2)$ the price that is output after running $A$ is equivalent to the price output when just running the auction without $b_i$. In other words, the resulting price that is charged is still $p$.

Therefore, no matter what $b$ does they will either win and pay $p$ or lose and not pay anything.

\subsection*{Part b}
We wish to prove that if all other bidders tell the truth then bidder $i$'s best response is to make a winning bid.

A bidder's utility is equal to $v - p$, where $v$ is the value of the item to a particular bidder and $p$ is the price they pay for that item. If $v_i > v_j$ for all bidders $j \neq i$, then we know that bidder $i$'s utility $v_i - p$ must also be the greatest.

We also know that if bidder $i$ wins their utility must be equal to $v_i - b_{A(\vec{b}_{-1}, -2)}$, since it is given that the winner is charged $b_{A(\vec{b}_{-1}, -2)}$. Additionally, whatever bid $b_i$ bidder $i$ makes must also be greater than $b_{A(\vec{b}_{-1}, -2)}$. Finally, because it is assumed that all other bidders $j \neq i$ tell the truth the price they would pay is equal to $v_{A(\vec{b}_{-1}, -2)}$. In other words, they are strictly incentivized to report the true value of the item to them as their bid (and so would end up paying $v_{A(\vec{b}_{-1}, -2)}$).

Since $v_i > v_j$ for all bidders $j \neq i$ and $b_i > b_{A(\vec{b}_{-1}, -2)}$, a winning bid for bidder $i$ will always be positive. Therefore, it is in bidder $i$'s best interest to make a winning bid.

\subsection*{Part c}

We wish to prove that error-prone second-price auction is not incentive compatible by providing an example where the second-highest bidder wins and is strictly worse off despite all voters telling the truth. Let us consider three bidders 1, 2, 3 and their associated bids:

\begin{center}
\begin{tabular}{|c|c|}
\hline
Bid   & Amount \\ \hline
$b_1$ & 4      \\ \hline
$b_2$ & 3      \\ \hline
$b_3$ & 1      \\ \hline
\end{tabular}
\end{center}

Even though $b_1$ is the largest bid bidder 2 will be selected as the winner because $b_2$ is exactly one less than $b_1$. Therefore, $b_2$ will be declared the winner and will be charged $b_{A(\vec{b}_{-1}, -2)}$. We then consider the remaining set of bids

\begin{center}
\begin{tabular}{|c|c|}
\hline
Bid   & Amount \\ \hline
$b_1$ & 4      \\ \hline
$b_3$ & 1      \\ \hline
\end{tabular}
\end{center}

The bid $b_1$ will be selected since it is more than one greater than the second-highest $b_3$. Therefore, the winner bidder 2 will be charged $b_1 = 4$. If we assume that everyone tells the truth then bidder 2's payoff must be $v_2 - p = 3 - 4 = -1$ since $v_2$ is taken as the true value of the item to them. Because bidder 2 received a strictly worse payoff even though all bidders told the truth, the auction is not incentive compatible.

\subsection*{Part d}

We wish to prove that error-prone second-price auction is not incentive compatible by providing an example where the highest bidder wins but the second-highest bidder would have been strictly happier by lying.

Let us consider three bidders 1, 2, 3 and their associated bids:

\begin{center}
\begin{tabular}{|c|c|}
\hline
Bid   & Amount \\ \hline
$b_1$ & 4      \\ \hline
$b_2$ & 3.5      \\ \hline
$b_3$ & 3      \\ \hline
\end{tabular}
\end{center}

Bidder 1 will be selected as the winner because the next highest bid, $b_2$, is not exactly one less than $b_1$. We then consider the remaining set of bids: 

\begin{center}
\begin{tabular}{|c|c|}
\hline
Bid   & Amount \\ \hline
$b_2$ & 3.5      \\ \hline
$b_3$ & 3      \\ \hline
\end{tabular}
\end{center}

This means that bidder 1 will be charged $b_2 = 3.5$. However we then consider the following alternate bids where bidder 2 chooses to lie:

\begin{center}
\begin{tabular}{|c|c|}
\hline
Bid   & Amount \\ \hline
$b_1$ & 4      \\ \hline
$b_2$ & 4.5      \\ \hline
$b_3$ & 3      \\ \hline
\end{tabular}
\end{center}

Here bidder 2 will be selected as the winner, and will end up paying $b_3 = 3$. Since $b_2 > b_3$, bidder 2 would have been strictly happier by lying.

\end{document}
